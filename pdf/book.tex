

	\documentclass[10pt,parskip=half,
	toc=sectionentrywithdots,
	bibliography=totocnumbered,
	captions=tableheading,numbers=noendperiod]{scrartcl}

    \usepackage[T1]{fontenc} % Nicer default font (+ math font) than Computer Modern for most use cases
    \usepackage{mathpazo}
    \usepackage{graphicx}
    \usepackage[skip=3pt]{caption}
    \usepackage{adjustbox} % Used to constrain images to a maximum size
    \usepackage[table]{xcolor} % Allow colors to be defined
    \usepackage{enumerate} % Needed for markdown enumerations to work
    \usepackage{amsmath} % Equations
    \usepackage{amssymb} % Equations
    \usepackage{textcomp} % defines textquotesingle
    % Hack from http://tex.stackexchange.com/a/47451/13684:
    \AtBeginDocument{%
        \def\PYZsq{\textquotesingle}% Upright quotes in Pygmentized code
    }
    \usepackage{upquote} % Upright quotes for verbatim code
    \usepackage{eurosym} % defines \euro
    \usepackage[mathletters]{ucs} % Extended unicode (utf-8) support
    \usepackage[utf8x]{inputenc} % Allow utf-8 characters in the tex document
    \usepackage{fancyvrb} % verbatim replacement that allows latex
    \usepackage{grffile} % extends the file name processing of package graphics
                         % to support a larger range
    % The hyperref package gives us a pdf with properly built
    % internal navigation ('pdf bookmarks' for the table of contents,
    % internal cross-reference links, web links for URLs, etc.)
    \usepackage{hyperref}
    \usepackage{longtable} % longtable support required by pandoc >1.10
    \usepackage{booktabs}  % table support for pandoc > 1.12.2
    \usepackage[inline]{enumitem} % IRkernel/repr support (it uses the enumerate* environment)
    \usepackage[normalem]{ulem} % ulem is needed to support strikethroughs (\sout)
                                % normalem makes italics be italics, not underlines

    \usepackage{translations}
	\usepackage{microtype} % improves the spacing between words and letters
	\usepackage{placeins} % placement of figures
    % could use \usepackage[section]{placeins} but placing in subsection in command section
	% Places the float at precisely the location in the LaTeX code (with H)
	\usepackage{float}
	\usepackage[colorinlistoftodos,obeyFinal,textwidth=.8in]{todonotes} % to mark to-dos
	% number figures, tables and equations by section
	\usepackage{chngcntr}
	% header/footer
	\usepackage[footsepline=0.25pt]{scrlayer-scrpage}

	% bibliography formatting
	\usepackage[numbers, square, super, sort&compress]{natbib}
	% hyperlink doi's
	\usepackage{doi}

    % define a code float
    \usepackage{newfloat} % to define a new float types
    \DeclareFloatingEnvironment[
        fileext=frm,placement={!ht},
        within=section,name=Code]{codecell}
    \DeclareFloatingEnvironment[
        fileext=frm,placement={!ht},
        within=section,name=Text]{textcell}
    \DeclareFloatingEnvironment[
        fileext=frm,placement={!ht},
        within=section,name=Text]{errorcell}

    \usepackage{listings} % a package for wrapping code in a box
    \usepackage[framemethod=tikz]{mdframed} % to fram code

% Pygments definitions

\makeatletter
\def\PY@reset{\let\PY@it=\relax \let\PY@bf=\relax%
    \let\PY@ul=\relax \let\PY@tc=\relax%
    \let\PY@bc=\relax \let\PY@ff=\relax}
\def\PY@tok#1{\csname PY@tok@#1\endcsname}
\def\PY@toks#1+{\ifx\relax#1\empty\else%
    \PY@tok{#1}\expandafter\PY@toks\fi}
\def\PY@do#1{\PY@bc{\PY@tc{\PY@ul{%
    \PY@it{\PY@bf{\PY@ff{#1}}}}}}}
\def\PY#1#2{\PY@reset\PY@toks#1+\relax+\PY@do{#2}}

\expandafter\def\csname PY@tok@w\endcsname{\def\PY@tc##1{\textcolor[rgb]{0.73,0.73,0.73}{##1}}}
\expandafter\def\csname PY@tok@c\endcsname{\let\PY@it=\textit\def\PY@tc##1{\textcolor[rgb]{0.25,0.50,0.50}{##1}}}
\expandafter\def\csname PY@tok@cp\endcsname{\def\PY@tc##1{\textcolor[rgb]{0.74,0.48,0.00}{##1}}}
\expandafter\def\csname PY@tok@k\endcsname{\let\PY@bf=\textbf\def\PY@tc##1{\textcolor[rgb]{0.00,0.50,0.00}{##1}}}
\expandafter\def\csname PY@tok@kp\endcsname{\def\PY@tc##1{\textcolor[rgb]{0.00,0.50,0.00}{##1}}}
\expandafter\def\csname PY@tok@kt\endcsname{\def\PY@tc##1{\textcolor[rgb]{0.69,0.00,0.25}{##1}}}
\expandafter\def\csname PY@tok@o\endcsname{\def\PY@tc##1{\textcolor[rgb]{0.40,0.40,0.40}{##1}}}
\expandafter\def\csname PY@tok@ow\endcsname{\let\PY@bf=\textbf\def\PY@tc##1{\textcolor[rgb]{0.67,0.13,1.00}{##1}}}
\expandafter\def\csname PY@tok@nb\endcsname{\def\PY@tc##1{\textcolor[rgb]{0.00,0.50,0.00}{##1}}}
\expandafter\def\csname PY@tok@nf\endcsname{\def\PY@tc##1{\textcolor[rgb]{0.00,0.00,1.00}{##1}}}
\expandafter\def\csname PY@tok@nc\endcsname{\let\PY@bf=\textbf\def\PY@tc##1{\textcolor[rgb]{0.00,0.00,1.00}{##1}}}
\expandafter\def\csname PY@tok@nn\endcsname{\let\PY@bf=\textbf\def\PY@tc##1{\textcolor[rgb]{0.00,0.00,1.00}{##1}}}
\expandafter\def\csname PY@tok@ne\endcsname{\let\PY@bf=\textbf\def\PY@tc##1{\textcolor[rgb]{0.82,0.25,0.23}{##1}}}
\expandafter\def\csname PY@tok@nv\endcsname{\def\PY@tc##1{\textcolor[rgb]{0.10,0.09,0.49}{##1}}}
\expandafter\def\csname PY@tok@no\endcsname{\def\PY@tc##1{\textcolor[rgb]{0.53,0.00,0.00}{##1}}}
\expandafter\def\csname PY@tok@nl\endcsname{\def\PY@tc##1{\textcolor[rgb]{0.63,0.63,0.00}{##1}}}
\expandafter\def\csname PY@tok@ni\endcsname{\let\PY@bf=\textbf\def\PY@tc##1{\textcolor[rgb]{0.60,0.60,0.60}{##1}}}
\expandafter\def\csname PY@tok@na\endcsname{\def\PY@tc##1{\textcolor[rgb]{0.49,0.56,0.16}{##1}}}
\expandafter\def\csname PY@tok@nt\endcsname{\let\PY@bf=\textbf\def\PY@tc##1{\textcolor[rgb]{0.00,0.50,0.00}{##1}}}
\expandafter\def\csname PY@tok@nd\endcsname{\def\PY@tc##1{\textcolor[rgb]{0.67,0.13,1.00}{##1}}}
\expandafter\def\csname PY@tok@s\endcsname{\def\PY@tc##1{\textcolor[rgb]{0.73,0.13,0.13}{##1}}}
\expandafter\def\csname PY@tok@sd\endcsname{\let\PY@it=\textit\def\PY@tc##1{\textcolor[rgb]{0.73,0.13,0.13}{##1}}}
\expandafter\def\csname PY@tok@si\endcsname{\let\PY@bf=\textbf\def\PY@tc##1{\textcolor[rgb]{0.73,0.40,0.53}{##1}}}
\expandafter\def\csname PY@tok@se\endcsname{\let\PY@bf=\textbf\def\PY@tc##1{\textcolor[rgb]{0.73,0.40,0.13}{##1}}}
\expandafter\def\csname PY@tok@sr\endcsname{\def\PY@tc##1{\textcolor[rgb]{0.73,0.40,0.53}{##1}}}
\expandafter\def\csname PY@tok@ss\endcsname{\def\PY@tc##1{\textcolor[rgb]{0.10,0.09,0.49}{##1}}}
\expandafter\def\csname PY@tok@sx\endcsname{\def\PY@tc##1{\textcolor[rgb]{0.00,0.50,0.00}{##1}}}
\expandafter\def\csname PY@tok@m\endcsname{\def\PY@tc##1{\textcolor[rgb]{0.40,0.40,0.40}{##1}}}
\expandafter\def\csname PY@tok@gh\endcsname{\let\PY@bf=\textbf\def\PY@tc##1{\textcolor[rgb]{0.00,0.00,0.50}{##1}}}
\expandafter\def\csname PY@tok@gu\endcsname{\let\PY@bf=\textbf\def\PY@tc##1{\textcolor[rgb]{0.50,0.00,0.50}{##1}}}
\expandafter\def\csname PY@tok@gd\endcsname{\def\PY@tc##1{\textcolor[rgb]{0.63,0.00,0.00}{##1}}}
\expandafter\def\csname PY@tok@gi\endcsname{\def\PY@tc##1{\textcolor[rgb]{0.00,0.63,0.00}{##1}}}
\expandafter\def\csname PY@tok@gr\endcsname{\def\PY@tc##1{\textcolor[rgb]{1.00,0.00,0.00}{##1}}}
\expandafter\def\csname PY@tok@ge\endcsname{\let\PY@it=\textit}
\expandafter\def\csname PY@tok@gs\endcsname{\let\PY@bf=\textbf}
\expandafter\def\csname PY@tok@gp\endcsname{\let\PY@bf=\textbf\def\PY@tc##1{\textcolor[rgb]{0.00,0.00,0.50}{##1}}}
\expandafter\def\csname PY@tok@go\endcsname{\def\PY@tc##1{\textcolor[rgb]{0.53,0.53,0.53}{##1}}}
\expandafter\def\csname PY@tok@gt\endcsname{\def\PY@tc##1{\textcolor[rgb]{0.00,0.27,0.87}{##1}}}
\expandafter\def\csname PY@tok@err\endcsname{\def\PY@bc##1{\setlength{\fboxsep}{0pt}\fcolorbox[rgb]{1.00,0.00,0.00}{1,1,1}{\strut ##1}}}
\expandafter\def\csname PY@tok@kc\endcsname{\let\PY@bf=\textbf\def\PY@tc##1{\textcolor[rgb]{0.00,0.50,0.00}{##1}}}
\expandafter\def\csname PY@tok@kd\endcsname{\let\PY@bf=\textbf\def\PY@tc##1{\textcolor[rgb]{0.00,0.50,0.00}{##1}}}
\expandafter\def\csname PY@tok@kn\endcsname{\let\PY@bf=\textbf\def\PY@tc##1{\textcolor[rgb]{0.00,0.50,0.00}{##1}}}
\expandafter\def\csname PY@tok@kr\endcsname{\let\PY@bf=\textbf\def\PY@tc##1{\textcolor[rgb]{0.00,0.50,0.00}{##1}}}
\expandafter\def\csname PY@tok@bp\endcsname{\def\PY@tc##1{\textcolor[rgb]{0.00,0.50,0.00}{##1}}}
\expandafter\def\csname PY@tok@fm\endcsname{\def\PY@tc##1{\textcolor[rgb]{0.00,0.00,1.00}{##1}}}
\expandafter\def\csname PY@tok@vc\endcsname{\def\PY@tc##1{\textcolor[rgb]{0.10,0.09,0.49}{##1}}}
\expandafter\def\csname PY@tok@vg\endcsname{\def\PY@tc##1{\textcolor[rgb]{0.10,0.09,0.49}{##1}}}
\expandafter\def\csname PY@tok@vi\endcsname{\def\PY@tc##1{\textcolor[rgb]{0.10,0.09,0.49}{##1}}}
\expandafter\def\csname PY@tok@vm\endcsname{\def\PY@tc##1{\textcolor[rgb]{0.10,0.09,0.49}{##1}}}
\expandafter\def\csname PY@tok@sa\endcsname{\def\PY@tc##1{\textcolor[rgb]{0.73,0.13,0.13}{##1}}}
\expandafter\def\csname PY@tok@sb\endcsname{\def\PY@tc##1{\textcolor[rgb]{0.73,0.13,0.13}{##1}}}
\expandafter\def\csname PY@tok@sc\endcsname{\def\PY@tc##1{\textcolor[rgb]{0.73,0.13,0.13}{##1}}}
\expandafter\def\csname PY@tok@dl\endcsname{\def\PY@tc##1{\textcolor[rgb]{0.73,0.13,0.13}{##1}}}
\expandafter\def\csname PY@tok@s2\endcsname{\def\PY@tc##1{\textcolor[rgb]{0.73,0.13,0.13}{##1}}}
\expandafter\def\csname PY@tok@sh\endcsname{\def\PY@tc##1{\textcolor[rgb]{0.73,0.13,0.13}{##1}}}
\expandafter\def\csname PY@tok@s1\endcsname{\def\PY@tc##1{\textcolor[rgb]{0.73,0.13,0.13}{##1}}}
\expandafter\def\csname PY@tok@mb\endcsname{\def\PY@tc##1{\textcolor[rgb]{0.40,0.40,0.40}{##1}}}
\expandafter\def\csname PY@tok@mf\endcsname{\def\PY@tc##1{\textcolor[rgb]{0.40,0.40,0.40}{##1}}}
\expandafter\def\csname PY@tok@mh\endcsname{\def\PY@tc##1{\textcolor[rgb]{0.40,0.40,0.40}{##1}}}
\expandafter\def\csname PY@tok@mi\endcsname{\def\PY@tc##1{\textcolor[rgb]{0.40,0.40,0.40}{##1}}}
\expandafter\def\csname PY@tok@il\endcsname{\def\PY@tc##1{\textcolor[rgb]{0.40,0.40,0.40}{##1}}}
\expandafter\def\csname PY@tok@mo\endcsname{\def\PY@tc##1{\textcolor[rgb]{0.40,0.40,0.40}{##1}}}
\expandafter\def\csname PY@tok@ch\endcsname{\let\PY@it=\textit\def\PY@tc##1{\textcolor[rgb]{0.25,0.50,0.50}{##1}}}
\expandafter\def\csname PY@tok@cm\endcsname{\let\PY@it=\textit\def\PY@tc##1{\textcolor[rgb]{0.25,0.50,0.50}{##1}}}
\expandafter\def\csname PY@tok@cpf\endcsname{\let\PY@it=\textit\def\PY@tc##1{\textcolor[rgb]{0.25,0.50,0.50}{##1}}}
\expandafter\def\csname PY@tok@c1\endcsname{\let\PY@it=\textit\def\PY@tc##1{\textcolor[rgb]{0.25,0.50,0.50}{##1}}}
\expandafter\def\csname PY@tok@cs\endcsname{\let\PY@it=\textit\def\PY@tc##1{\textcolor[rgb]{0.25,0.50,0.50}{##1}}}

\def\PYZbs{\char`\\}
\def\PYZus{\char`\_}
\def\PYZob{\char`\{}
\def\PYZcb{\char`\}}
\def\PYZca{\char`\^}
\def\PYZam{\char`\&}
\def\PYZlt{\char`\<}
\def\PYZgt{\char`\>}
\def\PYZsh{\char`\#}
\def\PYZpc{\char`\%}
\def\PYZdl{\char`\$}
\def\PYZhy{\char`\-}
\def\PYZsq{\char`\'}
\def\PYZdq{\char`\"}
\def\PYZti{\char`\~}
% for compatibility with earlier versions
\def\PYZat{@}
\def\PYZlb{[}
\def\PYZrb{]}
\makeatother

% ANSI colors
\definecolor{ansi-black}{HTML}{3E424D}
\definecolor{ansi-black-intense}{HTML}{282C36}
\definecolor{ansi-red}{HTML}{E75C58}
\definecolor{ansi-red-intense}{HTML}{B22B31}
\definecolor{ansi-green}{HTML}{00A250}
\definecolor{ansi-green-intense}{HTML}{007427}
\definecolor{ansi-yellow}{HTML}{DDB62B}
\definecolor{ansi-yellow-intense}{HTML}{B27D12}
\definecolor{ansi-blue}{HTML}{208FFB}
\definecolor{ansi-blue-intense}{HTML}{0065CA}
\definecolor{ansi-magenta}{HTML}{D160C4}
\definecolor{ansi-magenta-intense}{HTML}{A03196}
\definecolor{ansi-cyan}{HTML}{60C6C8}
\definecolor{ansi-cyan-intense}{HTML}{258F8F}
\definecolor{ansi-white}{HTML}{C5C1B4}
\definecolor{ansi-white-intense}{HTML}{A1A6B2}

% commands and environments needed by pandoc snippets
% extracted from the output of `pandoc -s`
\providecommand{\tightlist}{%
  \setlength{\itemsep}{0pt}\setlength{\parskip}{0pt}}
\DefineVerbatimEnvironment{Highlighting}{Verbatim}{commandchars=\\\{\}}
% Add ',fontsize=\small' for more characters per line
\newenvironment{Shaded}{}{}
\newcommand{\KeywordTok}[1]{\textcolor[rgb]{0.00,0.44,0.13}{\textbf{{#1}}}}
\newcommand{\DataTypeTok}[1]{\textcolor[rgb]{0.56,0.13,0.00}{{#1}}}
\newcommand{\DecValTok}[1]{\textcolor[rgb]{0.25,0.63,0.44}{{#1}}}
\newcommand{\BaseNTok}[1]{\textcolor[rgb]{0.25,0.63,0.44}{{#1}}}
\newcommand{\FloatTok}[1]{\textcolor[rgb]{0.25,0.63,0.44}{{#1}}}
\newcommand{\CharTok}[1]{\textcolor[rgb]{0.25,0.44,0.63}{{#1}}}
\newcommand{\StringTok}[1]{\textcolor[rgb]{0.25,0.44,0.63}{{#1}}}
\newcommand{\CommentTok}[1]{\textcolor[rgb]{0.38,0.63,0.69}{\textit{{#1}}}}
\newcommand{\OtherTok}[1]{\textcolor[rgb]{0.00,0.44,0.13}{{#1}}}
\newcommand{\AlertTok}[1]{\textcolor[rgb]{1.00,0.00,0.00}{\textbf{{#1}}}}
\newcommand{\FunctionTok}[1]{\textcolor[rgb]{0.02,0.16,0.49}{{#1}}}
\newcommand{\RegionMarkerTok}[1]{{#1}}
\newcommand{\ErrorTok}[1]{\textcolor[rgb]{1.00,0.00,0.00}{\textbf{{#1}}}}
\newcommand{\NormalTok}[1]{{#1}}

% Additional commands for more recent versions of Pandoc
\newcommand{\ConstantTok}[1]{\textcolor[rgb]{0.53,0.00,0.00}{{#1}}}
\newcommand{\SpecialCharTok}[1]{\textcolor[rgb]{0.25,0.44,0.63}{{#1}}}
\newcommand{\VerbatimStringTok}[1]{\textcolor[rgb]{0.25,0.44,0.63}{{#1}}}
\newcommand{\SpecialStringTok}[1]{\textcolor[rgb]{0.73,0.40,0.53}{{#1}}}
\newcommand{\ImportTok}[1]{{#1}}
\newcommand{\DocumentationTok}[1]{\textcolor[rgb]{0.73,0.13,0.13}{\textit{{#1}}}}
\newcommand{\AnnotationTok}[1]{\textcolor[rgb]{0.38,0.63,0.69}{\textbf{\textit{{#1}}}}}
\newcommand{\CommentVarTok}[1]{\textcolor[rgb]{0.38,0.63,0.69}{\textbf{\textit{{#1}}}}}
\newcommand{\VariableTok}[1]{\textcolor[rgb]{0.10,0.09,0.49}{{#1}}}
\newcommand{\ControlFlowTok}[1]{\textcolor[rgb]{0.00,0.44,0.13}{\textbf{{#1}}}}
\newcommand{\OperatorTok}[1]{\textcolor[rgb]{0.40,0.40,0.40}{{#1}}}
\newcommand{\BuiltInTok}[1]{{#1}}
\newcommand{\ExtensionTok}[1]{{#1}}
\newcommand{\PreprocessorTok}[1]{\textcolor[rgb]{0.74,0.48,0.00}{{#1}}}
\newcommand{\AttributeTok}[1]{\textcolor[rgb]{0.49,0.56,0.16}{{#1}}}
\newcommand{\InformationTok}[1]{\textcolor[rgb]{0.38,0.63,0.69}{\textbf{\textit{{#1}}}}}
\newcommand{\WarningTok}[1]{\textcolor[rgb]{0.38,0.63,0.69}{\textbf{\textit{{#1}}}}}

% Define a nice break command that doesn't care if a line doesn't already
% exist.
\def\br{\hspace*{\fill} \\* }

% Math Jax compatability definitions
\def\gt{>}
\def\lt{<}

    \setcounter{secnumdepth}{5}

    % Colors for the hyperref package
    \definecolor{urlcolor}{rgb}{0,.145,.698}
    \definecolor{linkcolor}{rgb}{.71,0.21,0.01}
    \definecolor{citecolor}{rgb}{.12,.54,.11}

\DeclareTranslationFallback{Author}{Author}
\DeclareTranslation{Portuges}{Author}{Autor}

\DeclareTranslationFallback{List of Codes}{List of Codes}
\DeclareTranslation{Catalan}{List of Codes}{Llista de Codis}
\DeclareTranslation{Danish}{List of Codes}{Liste over Koder}
\DeclareTranslation{German}{List of Codes}{Liste der Codes}
\DeclareTranslation{Spanish}{List of Codes}{Lista de C\'{o}digos}
\DeclareTranslation{French}{List of Codes}{Liste des Codes}
\DeclareTranslation{Italian}{List of Codes}{Elenco dei Codici}
\DeclareTranslation{Dutch}{List of Codes}{Lijst van Codes}
\DeclareTranslation{Portuges}{List of Codes}{Lista de C\'{o}digos}

\DeclareTranslationFallback{Supervisors}{Supervisors}
\DeclareTranslation{Catalan}{Supervisors}{Supervisors}
\DeclareTranslation{Danish}{Supervisors}{Vejledere}
\DeclareTranslation{German}{Supervisors}{Vorgesetzten}
\DeclareTranslation{Spanish}{Supervisors}{Supervisores}
\DeclareTranslation{French}{Supervisors}{Superviseurs}
\DeclareTranslation{Italian}{Supervisors}{Le autorit\`{a} di vigilanza}
\DeclareTranslation{Dutch}{Supervisors}{supervisors}
\DeclareTranslation{Portuguese}{Supervisors}{Supervisores}

\definecolor{codegreen}{rgb}{0,0.6,0}
\definecolor{codegray}{rgb}{0.5,0.5,0.5}
\definecolor{codepurple}{rgb}{0.58,0,0.82}
\definecolor{backcolour}{rgb}{0.95,0.95,0.95}

\lstdefinestyle{mystyle}{
    commentstyle=\color{codegreen},
    keywordstyle=\color{magenta},
    numberstyle=\tiny\color{codegray},
    stringstyle=\color{codepurple},
    basicstyle=\ttfamily,
    breakatwhitespace=false,
    keepspaces=true,
    numbers=left,
    numbersep=10pt,
    showspaces=false,
    showstringspaces=false,
    showtabs=false,
    tabsize=2,
    breaklines=true,
    literate={\-}{}{0\discretionary{-}{}{-}},
  postbreak=\mbox{\textcolor{red}{$\hookrightarrow$}\space},
}

\lstset{style=mystyle}

\surroundwithmdframed[
  hidealllines=true,
  backgroundcolor=backcolour,
  innerleftmargin=0pt,
  innerrightmargin=0pt,
  innertopmargin=0pt,
  innerbottommargin=0pt]{lstlisting}

 % Used to adjust the document margins
\usepackage{geometry}
\geometry{tmargin=1in,bmargin=1in,lmargin=1in,rmargin=1in,
nohead,includefoot,footskip=25pt}
% you can use showframe option to check the margins visually

	% ensure new section starts on new page
	\addtokomafont{section}{\clearpage}

    % Prevent overflowing lines due to hard-to-break entities
    \sloppy

    % Setup hyperref package
    \hypersetup{
      breaklinks=true,  % so long urls are correctly broken across lines
      colorlinks=true,
      urlcolor=urlcolor,
      linkcolor=linkcolor,
      citecolor=citecolor,
      }

    % ensure figures are placed within subsections
    \makeatletter
    \AtBeginDocument{%
      \expandafter\renewcommand\expandafter\subsection\expandafter
        {\expandafter\@fb@secFB\subsection}%
      \newcommand\@fb@secFB{\FloatBarrier
        \gdef\@fb@afterHHook{\@fb@topbarrier \gdef\@fb@afterHHook{}}}%
      \g@addto@macro\@afterheading{\@fb@afterHHook}%
      \gdef\@fb@afterHHook{}%
    }
    \makeatother

	% number figures, tables and equations by section
	\usepackage{chngcntr}
	\counterwithout{figure}{section}
	\counterwithout{table}{section}
	\counterwithout{equation}{section}
	\makeatletter
	\@addtoreset{table}{section}
	\@addtoreset{figure}{section}
	\@addtoreset{equation}{section}
	\makeatother
	\renewcommand\thetable{\thesection.\arabic{table}}
	\renewcommand\thefigure{\thesection.\arabic{figure}}
	\renewcommand\theequation{\thesection.\arabic{equation}}

        % set global options for float placement
        \makeatletter
          \providecommand*\setfloatlocations[2]{\@namedef{fps@#1}{#2}}
        \makeatother

    % align captions to left (indented)
	\captionsetup{justification=raggedright,
	singlelinecheck=false,format=hang,labelfont={it,bf}}

	% shift footer down so space between separation line
	\ModifyLayer[addvoffset=.6ex]{scrheadings.foot.odd}
	\ModifyLayer[addvoffset=.6ex]{scrheadings.foot.even}
	\ModifyLayer[addvoffset=.6ex]{scrheadings.foot.oneside}
	\ModifyLayer[addvoffset=.6ex]{plain.scrheadings.foot.odd}
	\ModifyLayer[addvoffset=.6ex]{plain.scrheadings.foot.even}
	\ModifyLayer[addvoffset=.6ex]{plain.scrheadings.foot.oneside}
	\pagestyle{scrheadings}
	\clearscrheadfoot{}
	\ifoot{\leftmark}
	\renewcommand{\sectionmark}[1]{\markleft{\thesection\ #1}}
	\ofoot{\pagemark}
	\cfoot{}

% clereref must be loaded after anything that changes the referencing system
\usepackage{cleveref}
\creflabelformat{equation}{#2#1#3}

% make the code float work with cleverref
\crefname{codecell}{code}{codes}
\Crefname{codecell}{code}{codes}
% make the text float work with cleverref
\crefname{textcell}{text}{texts}
\Crefname{textcell}{text}{texts}
% make the text float work with cleverref
\crefname{errorcell}{error}{errors}
\Crefname{errorcell}{error}{errors}

	\begin{document}

		\begin{titlepage}

	\begin{center}

	\vspace*{1cm}

	\Huge\textbf{Generating Software Tests}

	\vspace{0.5cm}\LARGE{for security testing}

	\vspace{1.5cm}

	\begin{minipage}{0.8\textwidth}
		\begin{center}
		\begin{minipage}{0.39\textwidth}
		\begin{flushleft} \Large
		\emph{\GetTranslation{Author}:}\\Andreas Zeller, Rahul Gopinath, Gordon Fraser, and Christian Holler\\
		\end{flushleft}
		\end{minipage}
		\hspace{\fill}
		\begin{minipage}{0.39\textwidth}
		\begin{flushright} \Large
		\end{flushright}
		\end{minipage}
		\end{center}
	\end{minipage}

	\vfill

	\begin{minipage}{0.8\textwidth}
	\begin{center}
	\end{center}
	\end{minipage}

	\vspace{0.8cm}

	\vspace{0.4cm}

	\today

	\end{center}
	\end{titlepage}

		\begingroup
    \let\cleardoublepage\relax
    \let\clearpage\relax\tableofcontents
    \endgroup

\vfill

\textbf{Generating Software Tests}, by Andreas Zeller, Rahul Gopinath,
Gordon Fraser, and Christian Holler.

Copyright © 2018 by the authors; all rights reserved.

\section{Fuzzing 101}\label{fuzzing-101}

We'll start with a simple fuzzer. The idea is to produce random
characters, adding them to a buffer string variable (\texttt{out}), and
finally returning the string.

\subsection{A Simple Fuzzer}\label{a-simple-fuzzer}

This implementation uses the following Python features and functions:

\begin{itemize}
\tightlist
\item
  \texttt{random.randrange(start,\ end)} - return a random number
  {[}\texttt{start}, \texttt{end}{]}
\item
  \texttt{range(start,\ end)} - create a list with integers from
  \texttt{start} to \texttt{end}. Typically used in iterations.
\item
  \texttt{for\ elem\ in\ list:\ body} executes \texttt{body} in a loop
  with \texttt{elem} taking each value from \texttt{list}.
\item
  \texttt{for\ i\ in\ range(start,\ end):\ body} executes \texttt{body}
  in a loop with \texttt{i} from \texttt{start} to \texttt{end} - 1.
\item
  \texttt{chr(n)} - return a character with ASCII code \texttt{n}
\end{itemize}

\begin{lstlisting}[language=Python,numbers=left,xleftmargin=20pt,xrightmargin=5pt,belowskip=5pt,aboveskip=5pt]
import gstbook
\end{lstlisting}

\begin{lstlisting}[language=Python,numbers=left,xleftmargin=20pt,xrightmargin=5pt,belowskip=5pt,aboveskip=5pt]
import random
\end{lstlisting}

\begin{lstlisting}[language=Python,numbers=left,xleftmargin=20pt,xrightmargin=5pt,belowskip=5pt,aboveskip=5pt]
# We set a specific seed to get the same inputs each time
random.seed(53727895348829)
\end{lstlisting}

\begin{lstlisting}[language=Python,numbers=left,xleftmargin=20pt,xrightmargin=5pt,belowskip=5pt,aboveskip=5pt]
def fuzzer(max_length=100, char_start=32, char_range=32):
    """A string of up to `max_length` characters 
       in the range [`char_start`, `char_start` + `char_range`]"""
    string_length = random.randrange(0, max_length)
    out = ""
    for i in range(0, string_length):
        out += chr(random.randrange(char_start, char_start + char_range))
    return out
\end{lstlisting}

With its default arguments, the \texttt{fuzzer()} function returns a
string of random characters:

\begin{lstlisting}[language=Python,numbers=left,xleftmargin=20pt,xrightmargin=5pt,belowskip=5pt,aboveskip=5pt]
fuzzer(1000, 64, 32)
\end{lstlisting}

\begin{lstlisting}[language={},postbreak={},numbers=none,xrightmargin=7pt,breakindent=0pt,aboveskip=5pt,belowskip=5pt]
'VNCETHCT_DLORCXT[I@F[[QIFTU]YSVQXH]RV^BPFLW^M@IS[DCVHCPXPU]NL@DIQWIZ]BQUL^LOXQRCYCE^WC^PCJRUAUNXG\\MAKTBGFRI_MRZ]NZJLABPNZC@FULS__]KMBJKFYXMH@XB]XGXEEFHQ^VNVSLLA_HF]_^W]T@RJO^N]UQRZ^[]JLUBXBHJG[_D@NZ@UA@KY@IQLP\\YVJES_V^LZMTX@ZJGCBFXZVNS^@HITIQMEZHIGKGQWDFZ[HL_AENNDK[]LLK[I@GWH^WH]IRMJGGAJJZCGOSQRLZVF\\RVYVHBFZGOU_O]GHTY\\\\MDYZGYIHL^UEYERY\\@HVOJGX]]MPGSWTHBZDWXOIFMEKTKNE]CX[J]WOGGI[NDYGL_XDLEC@JRJPOWMOMXZPOAO_@GICMI^VADUMHXHP\\Y^SUJRXCPQ@HTQBXEAV_YLH_EFKM@L@AGVFAIZZY[\\LXO@]DDUDIRQYLI^YL\\TP^VL^XE]P^BA_@E^LWXG\\XVJGE_P\\DJW][W\\JJGV[YWCE^]H]QKX_UZ^NQAGDFS[RVAKH\\@^_QIDM@BDZMPTHQB_LLURC]SASN\\[NIOUFYMLCE[L^^GTJMST[SQWUR\\UHIY_[QOT'
\end{lstlisting}

\subsection{Fuzzing Alphabets}\label{fuzzing-alphabets}

We can also have \texttt{fuzzer()} produce a series of upercase letters.
We use \texttt{ord(c)} to return the ASCII code of the character
\texttt{c}.

\begin{lstlisting}[language=Python,numbers=left,xleftmargin=20pt,xrightmargin=5pt,belowskip=5pt,aboveskip=5pt]
fuzzer(100, ord('A'), 26)
\end{lstlisting}

\begin{lstlisting}[language={},postbreak={},numbers=none,xrightmargin=7pt,breakindent=0pt,aboveskip=5pt,belowskip=5pt]
'NCSJGSPARSBYUGBJFMMZPRMGEMBWHYBNBLODXQUKEZWUINIXTFSKZAOLXNOURBUJWBXHXZBNYHNGHGYIUYBNUSSFJGY'
\end{lstlisting}

We can also have it produce a series of digits:

\begin{lstlisting}[language=Python,numbers=left,xleftmargin=20pt,xrightmargin=5pt,belowskip=5pt,aboveskip=5pt]
fuzzer(100, ord('0'), 10)
\end{lstlisting}

\begin{lstlisting}[language={},postbreak={},numbers=none,xrightmargin=7pt,breakindent=0pt,aboveskip=5pt,belowskip=5pt]
'96735410455640'
\end{lstlisting}

My personal favorite, though, is still simply to have \texttt{fuzzer()}
produce a long list of garbage...

\begin{lstlisting}[language=Python,numbers=left,xleftmargin=20pt,xrightmargin=5pt,belowskip=5pt,aboveskip=5pt]
fuzzer(2000, 32, 96)
\end{lstlisting}

\begin{lstlisting}[language={},postbreak={},numbers=none,xrightmargin=7pt,breakindent=0pt,aboveskip=5pt,belowskip=5pt]
'\x7f\\mP89i^s::?qac{2b{[@H;_Ub Yvw%^5_KMux|kJP\\Lq"*X"Q?+g9*/Ttt8@e/RvCP5BIGnU4m;|yD}-GnxJ(ALufq#1h1VXxMCX\x7fUM+"^M6:y23>:WMKhwau;Dn<V%Qz]Ln:`QN s`NTS2#6^j$,~3Mpm$Ot{C5o&<t:7n|ml}XE-U%j51"Vbl;>TK$Dv+%C"JfUa)NETSQtmM+f^@(G32+(vB`o@W|RU6a$21xcqD&\\DlVz|ej!\x7f-1.\\Iz+o6J:tncz\x7fw2-LR4_(TQQ`8BM>kUMcC\x7f<V;ZP[|ERHIC"(\'r@yVO!HwsN_TAMs8oK!}Zcly|$`qk[iGzR }o~i@?btW4%<&?KP"2F0+6pq2d1\x7f:xd>CU[U?Wb)o&U0O{,zp~9DfH/.\x7f/amvvJr0NT@H_>[w5Gkgk*kl.;u}bN/7</k&!u5XZ&A3w"W_{*JbPyP6=a[AP>(8%B]&r\'x{>%Guodp8fD1JU*+~_SXFm}5Lo<X&?}ho.ok:2=FXblz:|~1lMd4F.R&i,TE^-cy%3D!}CHudEO5TUb/_7!)W~a=ng1ZbMrn|@ jK>#f;%j$!6[, Pm[JUa14KHeO7ie}g7e,$vHj6A`dim%AV|yChKC;AB<%dB,OB$ME_X|lI?Hx5o=cn4yAq)Ge/BC:'
\end{lstlisting}

... and to feed this into some function or program.

\begin{lstlisting}[language=Python,numbers=left,xleftmargin=20pt,xrightmargin=5pt,belowskip=5pt,aboveskip=5pt]
if __name__ == "__main__":
    x = int(fuzzer(100, ord('0'), 10))
    print(x)
\end{lstlisting}

\begin{lstlisting}[language={},postbreak={},numbers=none,xrightmargin=7pt,belowskip=5pt,aboveskip=5pt,breakindent=0pt]
418655427319022234162993347851489442135051633291608613469536946573

\end{lstlisting}

\subsection{Background}\label{background}

Here's a citation, just for the fun: \cite{purdom1972}

\section{Getting Coverage}\label{getting-coverage}

Let us obtain the coverage of a simple function. To this end, we use a
trace function that tracks each line executed.

\subsection{Tracing Functions}\label{tracing-functions}

This implementation uses the following Python features and functions:

\begin{itemize}
\tightlist
\item
  \texttt{sys.settrace(f)} - set \texttt{f()} as tracing function, to be
  called for each line.
\end{itemize}

As \texttt{f}, we define a function \texttt{traceit()}, which accesses
the current function name and current line number, as shown below, and
store this in a global map \texttt{coverage}. Note that depending on
your Python setting, \texttt{traceit()} may also be called for other,
internal functions, and hence, these may also be covered in
\texttt{coverage}.

\begin{lstlisting}[language=Python,numbers=left,xleftmargin=20pt,xrightmargin=5pt,belowskip=5pt,aboveskip=5pt]
import sys

# Where we store the coverage - a mapping of function names to sets of lines
coverage = {}

# Now, some dynamic analysis
def traceit(frame, event, arg):
    global coverage
    if event == "line":
        function_name = frame.f_code.co_name
        lineno = frame.f_lineno
        if not function_name in coverage:
            coverage[function_name] = set()
        coverage[function_name].add(lineno)
    return traceit
\end{lstlisting}

To test our coverage setting, we use the \texttt{cgi\_decode()} function
(after Pezze and Young), which takes a CGI-encoded string (say, "a+b")
and returns the decoded variant (say, "a b").

\begin{lstlisting}[language=Python,numbers=left,xleftmargin=20pt,xrightmargin=5pt,belowskip=5pt,aboveskip=5pt]
def cgi_decode(s):
    """Decode the CGI-encoded string `s`:
       * replace "+" by " "
       * replace "%xx" by the character with hex number xx.
       Return the decoded string, or None for invalid inputs."""

    # Mapping of hex digits to their integer values
    hex_values = {
        '0': 0,
        '1': 1,
        '2': 2,
        '3': 3,
        '4': 4,
        '5': 5,
        '6': 6,
        '7': 7,
        '8': 8,
        '9': 9,
        'a': 10,
        'b': 11,
        'c': 12,
        'd': 13,
        'e': 14,
        'f': 15,
        'A': 10,
        'B': 11,
        'C': 12,
        'D': 13,
        'E': 14,
        'F': 15,
    }

    t = ""
    i = 0
    while i < len(s):
        c = s[i]
        if c == '+':
            t = t + ' '
        elif c == '%':
            digit_high = s[i + 1]
            digit_low  = s[i + 2]
            i = i + 2
            if digit_high in hex_values and digit_low in hex_values:
                v = hex_values[digit_high] * 16 + hex_values[digit_low]
                t = t + chr(v)
            else:
                return None
        else:
            t = t + c
        i = i + 1
    return t

# A few unit tests    
assert cgi_decode('+') == ' '
assert cgi_decode('%20') == ' '
assert cgi_decode('abc') == 'abc'
assert cgi_decode('%?a') is None

\end{lstlisting}

We can now obtain the coverage for a run:

\begin{lstlisting}[language=Python,numbers=left,xleftmargin=20pt,xrightmargin=5pt,belowskip=5pt,aboveskip=5pt]
coverage = {}
sys.settrace(traceit)
x = cgi_decode('abc')
sys.settrace(None)
abc_coverage = coverage['cgi_decode']
print(abc_coverage)
\end{lstlisting}

\begin{lstlisting}[language={},postbreak={},numbers=none,xrightmargin=7pt,belowskip=5pt,aboveskip=5pt,breakindent=0pt]
{9, 10, 11, 12, 13, 14, 15, 16, 17, 18, 19, 20, 21, 22, 23, 24, 25, 26, 27, 28, 29, 30, 33, 34, 35, 36, 37, 39, 49, 50, 51}

\end{lstlisting}

\subsection{Comparing Coverages}\label{comparing-coverages}

Different inputs cause different coverages:

\begin{lstlisting}[language=Python,numbers=left,xleftmargin=20pt,xrightmargin=5pt,belowskip=5pt,aboveskip=5pt]
coverage = {}
sys.settrace(traceit)
x = cgi_decode('a+b')
sys.settrace(None)
a_plus_b_coverage = coverage['cgi_decode']
print(a_plus_b_coverage)
\end{lstlisting}

\begin{lstlisting}[language={},postbreak={},numbers=none,xrightmargin=7pt,belowskip=5pt,aboveskip=5pt,breakindent=0pt]
{9, 10, 11, 12, 13, 14, 15, 16, 17, 18, 19, 20, 21, 22, 23, 24, 25, 26, 27, 28, 29, 30, 33, 34, 35, 36, 37, 38, 39, 49, 50, 51}

\end{lstlisting}

We see that the input \texttt{"a+b"} covers one more line than
\texttt{"abc"}, namely the code line where
\texttt{\textquotesingle{}+\textquotesingle{}} is processed:

\begin{lstlisting}[language=Python,numbers=left,xleftmargin=20pt,xrightmargin=5pt,belowskip=5pt,aboveskip=5pt]
print(a_plus_b_coverage - abc_coverage)
\end{lstlisting}

\begin{lstlisting}[language={},postbreak={},numbers=none,xrightmargin=7pt,belowskip=5pt,aboveskip=5pt,breakindent=0pt]
{38}

\end{lstlisting}

With this tool, we can now

\begin{itemize}
\tightlist
\item
  \emph{assess} coverage to evaluate how good our test generators are
\item
  \emph{leverage} coverage to guide our test generators towards
  uncovered code.
\end{itemize}

\subsection{Exercises}\label{exercises}

\begin{enumerate}
\def\labelenumi{\arabic{enumi}.}
\tightlist
\item
  Measure the \emph{branch coverage} in a function by storing
  \emph{pairs} of lines executed one after the other.
\item
  Generalize your solution such that it also works with function calls
  and returns, including recursive calls. See the documentation of
  \texttt{sys.settrace()} to track calling events.
\end{enumerate}

% sort citations by order of first appearance
\bibliographystyle{unsrtnat}
\bibliography{book_files/gstbook.bib}

	\end{document}

